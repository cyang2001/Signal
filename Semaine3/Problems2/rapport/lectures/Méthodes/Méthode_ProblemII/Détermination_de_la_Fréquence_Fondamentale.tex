Le décodage des caractéristiques de chaque note s'effectue en calculant la puissance moyenne, la fréquence fondamentale ($f_0$), la fréquence de la plus haute harmonique ($f_h$), et le nombre d'harmoniques dans la bande de fréquences.La détection précise de la fréquence fondamentale ($f_0$) est essentielle dans l'analyse des signaux sonores, notamment pour l'étude des notes musicales. Ces informations sont extraites à partir du signal audio, utilisant diverses fonctions telles que le calcul de la moyenne de puissance, la fréquence fondamentale par auto-corrélation, et le comptage des harmoniques. Les résultats sont affichés pour chaque note, fournissant une caractérisation détaillée des éléments musicaux contenus chaque fichier audio ainsi que la note et l'octave associée à la fréquence. 

L'ensemble de ces étapes contribue à élaborer un algorithme complet de détection et de caractérisation des notes de musique, avec une option facultative pour estimer la direction de la source sonore sur des signaux acquis par deux micros différents. 

Dans notre étude, nous avons appliqué deux méthodes distinctes pour calculer cette fréquence fondamentale, puis nous les avons comparées afin de vérifier leur exactitude.

La première méthode s'appuie sur la détection de la fréquence qui présente la densité de puissance la plus élevée après avoir effectué une transformation de Fourier rapide (FFT). Cette approche est basée sur l'hypothèse que, dans la majorité des cas, la fréquence fondamentale correspond à la fréquence la plus basse ayant la plus grande densité de puissance.
\begin{equation}
    f_0 = \arg\max_f(\frac{2}{N}|FFT(signal)|)
\end{equation}

Quant à la seconde méthode, nous utilisons le principe de l'auto-corrélation dans le domaine temporel. Cette technique permet de détecter les répétitions périodiques au sein du signal, ce qui facilite l'identification de la fréquence fondamentale, même en présence de harmoniques complexes ou de bruit.
\begin{equation}
    f_0 = \frac{f_s}{Lag_{peak}}
\end{equation}
Où: 
\begin{itemize}
    \item $f_0$ est la fréquence fondamentale  
    \item $f_s$ est la fréquence d'échantillonnage 
    \item $Lag_{peak}$ est le temps de retard correspondant au premier pic significatif de la fonction d'auto-corrélation.
\end{itemize}

Ces méthodes contribuent à élaborer un algorithme complet de détection et de caractérisation des notes de musique. En combinant et en comparant les résultats obtenus par ces deux méthodes, nous pouvons augmenter la fiabilité de notre analyse et assurer une détection plus précise de la fréquence fondamentale, ce qui est crucial pour une compréhension approfondie des propriétés acoustiques des notes analysées.
