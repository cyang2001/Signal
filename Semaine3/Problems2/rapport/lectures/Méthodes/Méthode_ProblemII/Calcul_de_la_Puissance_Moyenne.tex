
En cette partie, nous allons calculer la puissance moyenne de la partie des notes précédemment isolées.

\begin{equation}
    \overline{P} = \frac{1}{t_{\text{end}} f_s - t_{\text{start}} f_s} \sum_{n=t_{\text{start}} f_s}^{t_{\text{end}} f_s} P(n)^2
\end{equation}


Où :
\begin{itemize}
  \item \( f_s \) est la fréquence d'échantillonnage
  \item \( t_{\text{start}} \) est l'heure de début
  \item \( t_{\text{end}} \) est l'heure de fin
  \item \( n \) est l'indice de l'échantillon
\end{itemize}

La puissance moyenne en dBm peut alors être calculée en utilisant :
\begin{equation}
    \text{avgPower}_{\text{dBm}} = 10 \cdot \log_{10}\left(\frac{\overline{P}}{0.001}\right)
\end{equation}

