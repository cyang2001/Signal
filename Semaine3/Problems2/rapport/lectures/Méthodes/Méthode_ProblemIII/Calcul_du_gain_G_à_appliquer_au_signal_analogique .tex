Reprenons la formule du problème I :
\begin{equation}
    P_{dBm} = 20\log(S_{REF}*10^\frac{S}{20}*P_{REF}*10^\frac{P_{SPL}}{20}*10^\frac{G}{20})
\end{equation}
pour la simplifier : 
\begin{equation}
    P_{dBm} = 20\log(S_{REF}) + S + 20\log(P_{REF}) + P_{SPL} + G + 30
\end{equation}
avec 
\begin{itemize}
    \item $S_{REF} = 1\frac{V}{P_a}$
    \item $P_{REF} = 20\mu P_a$
    \item $S = -47 \text{DBV}$
\end{itemize}
donc nous avons :
\begin{equation}
    P_{dbm} = 10log(V_a^2 * 1000) = 30
\end{equation}
(comme la dynamique après amplification est [-1V,+1V], donc $V_a = 1V$)
; nous obtenons :
\begin{equation}
    G = P_{dBm} - 20\log(1) - 20\log(20*10^{-6}) - S - P_{SPL} - 30 
\end{equation}
soit,
\tcbox[colback=yellow!20, colframe=yellow, boxrule=2pt]{$G = P_{dBm} - S - P_{SPL} - (30 + 20log(20*10^{-6}))$}
Avec l'application numérique, nous trouvons notre gain.
\tcbox[colback=green!20, colframe=green, boxrule=2pt]{$G = 30 + 47 - 130 -(30 + 20\log(20*10^{-6})) \approx 11$}
