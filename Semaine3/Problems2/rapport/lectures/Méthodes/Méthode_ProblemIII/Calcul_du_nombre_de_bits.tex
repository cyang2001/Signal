Pour ce faire, nous utiliserons la formule du rapport signal bruit SNR qui est définie comme :
\begin{equation}
SNR = 10 \log \left( \frac{P_s}{P_b} \right).
\end{equation}
Avec un SNR d'au moins \( 40 \, \text{dB} \), nous déterminons d'abord la puissance du signal utile \( P_s \) puis la puissance du bruit \( P_b \), ce qui nous permet de calculer le nombre de bits.

Reprenons la formule du gain pour déterminer la puissance en dBm :
\begin{equation}
G = P_{\text{dBm}} - S - P_{\text{SPL}} - \left( 30 + 20 \log \left( 20 \times 10^{-6} \right) \right).
\end{equation}
Ce qui nous donne alors :
\begin{equation}
P_{\text{dBm}} = G + S + P_{\text{SPL}} + 30 + 20 \log(P_{\text{ref}}).
\end{equation}

Avec \( P_{\text{SPL}} = 60 \, \text{dB SPL} \) et les mêmes valeurs que pour la partie 3.2.1, nous avons :
\begin{equation}
P_{\text{dBm}} = 11 - 47 + 60 + 30 + 20 \log(20 \times 10^{-6}) \quad \text{donc} \quad P_{\text{dBm}} = -40 \, \text{dBm}.
\end{equation}
\\

Pour convertir la puissance dBm en Watt, appliquons la formule :
\begin{equation}
P_W = 10\log{\left( \frac{P_{\text{dBm}}}{10^{-3}}  \right)} \quad \text{qui peut également être exprimée comme $P_{dBm} = 10\log(1000P_w)$ }.
\end{equation}
En isolant $P_w$, nous obtenons :
\begin{equation}
P_W = 10^{\left( \frac{P_{\text{dBm}}}{10^{-3}}  \right)} \quad \text{ce qui nous donne} \quad P_W = 10^{-7} \, \text{Watt}.
\end{equation}

De plus, nous avons :
\begin{equation}
SNR = 10 \log \left( \frac{P_S}{P_b} \right),
\end{equation}
donc
\begin{equation}
P_b = \frac{P_S}{10^{\frac{SNR}{10}}}.
\end{equation}
La puissance moyenne du bruit de quantification \( P_b \) est donc :
\begin{equation}
P_b = \frac{q^2}{12},
\end{equation}
soit,
\begin{equation}
q^2 = \frac{12 \times P_S}{10^{\frac{SNR}{10}}}.
\end{equation}
De plus, \( q \) est donné par :
\begin{equation}
q = \frac{2A}{2^b},
\end{equation}
donc
\begin{equation}
\left( \frac{2A}{2^b} \right)^2 = \frac{12 \times P_S}{10^{\frac{SNR}{10}}},
\end{equation}
et
\begin{equation}
A^2 \times 2^{2(1-b)} = \frac{12 \times P_S}{10^{\frac{SNR}{10}}}.
\end{equation}
Ainsi, nous avons :
\begin{equation}
2^{2(1-b)} = \frac{12 \times P_S}{A^2 \times 10^{\frac{SNR}{10}}},
\end{equation}
et en terminant les calculs :
\begin{equation}
2(1-b) = \log_2 \left( \frac{12 \times P_S}{A^2 \times 10^{\frac{SNR}{10}}} \right),
\end{equation}
puis
\tcbox[colback=green!20, colframe=green, boxrule=2pt]{$b = 1 - \frac{\log_2 \left( \frac{12 \times P_S}{A^2 \times 10^{\frac{SNR}{10}}} \right)}{2}$}
Avec A = 1V, SNR = 40 dB, et $P_s = 10 ^{-7} W$, nous obtenons b $\approx 17.47.$
\\
Nous savons que les bits prennent des valeurs entières, ainsi, nous avons donc \tcbox[colback=green!20, colframe=green, boxrule=2pt]{18 bits.}

