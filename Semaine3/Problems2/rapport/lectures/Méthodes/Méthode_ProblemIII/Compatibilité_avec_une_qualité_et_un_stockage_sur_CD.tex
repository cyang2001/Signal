La fréquence d'échantillonnage $F_e$ pour un CD standard est déjà établie à 44 100 Hz, tout comme le nombre de bits b qui est 16 bits et l'enregistrement sur CD se fait en stéréo.\footnote{Cours de cycle préparatoire associé avec Stanislas}
\\
Ainsi, \\
\tcbox[colback=yellow!20, colframe=yellow, boxrule=2pt]{$D = 44.1*10^3*16*2 = 1 411 200 
 bit/s$}

La capacité $C_{CD}$ est exprimée par $C_CD = D * d$, où D représente le nombre binaire et d la durée que nous avons sélectionnée, correspondant à 1 heure, soit 3 600 secondes.
\\
En effectuant une division par 8 pour obtenir le résultat en octet, la capacité $C_CD$ se calcule comme suit : \tcbox[colback=green!20, colframe=green, boxrule=2pt]{$C_CD = \frac{1 411 200*3600}{8} \approx 635 M_o$}

Ainsi, la capacité de stockage nécessaire pour enregistrer une heure d'audio en stéréo sur un CD standard est d'environ 635 $M_o$.
\\

Par conséquent, avec nos paramètres choisis, l'enregistrement \uline{n'est pas conforme aux spécifications d'un CD audio standard}, \textbf{si on est dans le cas où nous sommes en stéréo}. Il pourrait être envisageable d'explorer d'autres supports ou formats offrant une flexibilité plus grande pour répondre à des exigences spécifiques en termes de bande passante et de qualité audio. Sinon, dans le cas où \textbf{nous sommes en mono}, \uline{cela est conforme aux spécifications d'un CD audio standard}.

