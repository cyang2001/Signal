Pour récupérer de la musique avec fidélité, il convient de conserver le timbre musical, autrement dit, garder le plus d'harmoniques possibles, et pas uniquement le fondamental. Autrement dit, plus la fréquence d'échantillonnage est élevée, et plus nous avons de chance de respecter le timbre \footnote{\href{https://fr.wikipedia.org/wiki/Gain_d\%C3\%A9cibel}{Cliquez ici pour accéder à la source sur le timbre}} d'un instrument. 
\\Prenons un orchestre symphonique possédant une harpe, la plus haute note jouée avec cet instrument est un sol dièse 7 (3322.4 Hz)\footnote{\href{http://hyperphysics.phy-astr.gsu.edu/hbase/Music/orchins.html}{Cliquez ici pour accéder à la source de la fréquence de sol dièse 7 d'une harpe}}. Notre bande passante est de 20 000 Hz, nous pouvons donc récupérer 5 harmoniques d'un sol dièse 7 plus la fondamentale (nous atteignons alors 19 934.4 Hz).
\\De plus, comme précisé dans la partie 3.2.4, notre fréquence d'échantillonnage doit être au moins deux fois plus grande pour éviter le repliement spectral, ainsi, nous pouvons estimer que nous respecterons le timbre des instruments de musique dans un orchestre symphonique, et récupérer suffisamment d'harmonique de cet orchestre. L'oreille humaine ne pouvant pas réellement distinguer les sons au-delà de 14 000 Hz, cela est suffisant.
\\De surcroît, nous avons calculé un nombre de bits de quantification égale à 18. Ce nombre de bits est suffisamment élevé pour réduire au plus possible le bruit de quantification. En vérité, si le CD reste peu aimé pour l'écoute de musique classique\footnote{\href{https://www.radiofrance.fr/franceculture/podcasts/hashtag/la-musique-classique-peut-elle-encore-resister-a-la-dematerialisation-7501004}{Cliquez ici pour accéder à la source sur la musique la classique}}, c'est plus par appréciation de l'unité des performances d'un orchestre, ou par le fait de ne pas aimer le support  lui-même, mais la qualité d'enregistrement ne semble pas un problème. Le CD standard a 16 bits de quantification, nous pouvons conclure que le timbre des instruments d'un orchestre sera respecté.

