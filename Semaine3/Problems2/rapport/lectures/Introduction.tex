On propose une méthode de détection et de catégorisation des notes, qui permet de classer l'instrument sur lequel la note est jouée (violon, flûte, piano) dont les caractéristiques sont les suivantes: 
\begin{itemize}
  \item $t_d,t_f$: instants de début et de fin de la note
  \item $P_dBm$: la puissance moyenne du signal en dBm.
  \item $f_0$: la fréquence fondamentale et le nom et l'octave de la note jouée(pas encore fini, manque le nom de la note et l'octave)
  \item $f_h$: la fréquence haute telle que [0,$f_h$] contient 99\% de la puissance
  \item $n_h$: le nombre d'harmoniques dans cette bande de fréquences
\end{itemize}
Les signaux sont en stéréo, du coup on moyenne les signaux. \\
Pour détecter le son utile, on détecte les plages de son de puissance supérieure à 1\%  de la puissance maximale (en W), puis ne garde que celles de durée supérieure à une seconde. \\
