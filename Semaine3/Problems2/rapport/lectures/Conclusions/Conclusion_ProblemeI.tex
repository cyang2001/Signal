Pour conclure, le problème 1 a été abordé à travers la conception d’un algorithme dédié à la détection de la présence ou de l’absence de signaux audio, enregistrant ainsi le bruit ambiant. Nos expérimentations ont débuté en utilisant des sons préalablement définis, fournis dans la question F, tels que des bruits de marteau-piqueur, de ville, et de jardin 1 et 2. L’analyse approfondie de ces sons a permis de classer les signaux en deux catégories distinctes : les "sons pénibles", identifiés en rouge, définis par une durée excédant à 1 seconde et une puissance en dBm égale ou supérieure à 8, et les "sons acceptables" distingués en bleu, caractérisés par une puissance en dBm inférieure à ce seuil. 
\\

Nos résultats présentent des points forts notables. La méthode de classification des signaux a démontré une clarté visuelle, facilitée par l'utilisation de couleurs, renforçant ainsi la compréhension des différentes catégories. De plus, la caractérisation détaillée de chaque signal fournit une base solide pour évaluer l'impact des sons sur l'environnement.
\\

Cependant, des points faibles nécessitent une attention particulière. La détermination du seuil de puissance pour les "sons pénibles" pourrait être améliorée pour refléter davantage la variabilité des environnements sonores. De plus, la limitation actuelle de nos expérimentations à des enregistrements préétablis soulève des questions sur la généralisation de notre algorithme à des situations en temps réel et dynamique.
\\

Pour perfectionner notre recherche, nous envisageons d'optimiser la calibration du seuil de puissance en tenant compte de la variabilité environnementale. L'extension de nos expérimentations à des conditions réelles d'environnement sonore permettra une validation plus approfondie de notre algorithme. Ces perspectives d'amélioration visent à renforcer la fiabilité et l'applicabilité de notre méthode dans des situations plus diversifiées et changeantes.
