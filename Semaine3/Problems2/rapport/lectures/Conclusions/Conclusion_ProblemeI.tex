Pour conclure, le problème 1 a été abordé à travers la conception d’un algorithme dédié à la détection de la présence ou de l’absence de signaux audio, enregistrant ainsi le bruit ambiant. Nos expérimentations ont débuté en utilisant des sons préalablement définis, fournis dans la question F, tels que des bruits de marteau piqueur, de ville, et de jardin 1 et 2. L’analyse approfondie de ces sons a permis de classer les signaux en deux catégories distinctes : les «  sons pénibles » identifiés en rouge, définis par une durée excédant à 1 seconde et une puissance en dBm égale ou supérieur à 8, et les « sons acceptables » distingués en bleu, caractérisés par une puissance en dBm inférieure à ce seuil. Ces résultats préliminaires offrent une base solide pour la poursuite de notre recherche, en vue d’applications futures de cet algorithme dans des conditions réelles d’environnement sonore.