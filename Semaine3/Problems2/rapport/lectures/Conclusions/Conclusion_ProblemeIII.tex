En conclusion, notre démarche vise à réaliser un enregistrement numérique à partir d'un microphone. Après avoir déterminé un gain optimal de 11 dB et un nombre de bits de 18, nous avons calculé une capacité de stockage estimée de 648 $M_o$ pour 1 heure d'audio en stéréo.
\\

Cependant, l'estimation de la capacité de stockage, à 648 $M_o$ pour une heure d'audio en stéréo, présente une légère divergence par rapport à la norme d'un CD standard, fixée à 635 $M_o$. Cette différence souligne la complexité de l'équilibre entre la qualité audio et la contrainte de stockage, ce qui pourrait rendre une telle configuration incompatible avec un CD standard.
\\

Il est important de noter toutefois que si l'enregistrement est effectué en mono, notre approche pourrait être compatible avec les spécifications d'un CD standard. Cela met en lumière l'importance de considérer la configuration audio choisie lors de l'évaluation de la compatibilité avec les normes de stockage existantes. De plus, la réflexion sur l'adoption d'un pas de quantification logarithmique a ouvert des perspectives innovantes pour optimiser la représentation des signaux, en particulier à faible amplitude, tout en minimisant les distorsions.
\\

Notre approche méthodologique, du choix de gain à la réflexion sur la quantification, établit une base solide pour des ajustements futurs visant à concilier au mieux qualité sonore et contraintes de stockages dans des contextes spécifiques.