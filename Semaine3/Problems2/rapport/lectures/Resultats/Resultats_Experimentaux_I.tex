Après avoir appliqué la fonction du problème aux signaux Jardin01.mp3, Jardin02.mp3, Ville01.mp3 et MarteauPiqueur01.mp3, voici ce que nous obtenons :
\\
\\
\textbf{Légende :}
\\
\textcolor{red}{Rouge :} dépasse la durée et la puissance seuil
\\
\textcolor{blue}{Bleu : }dépasse la puissance seuil, mais plus petit que dt

\begin{figure}[htb]
\subfigure[Application sur le signal Jardin01]{
\includegraphics[width = 0.45\textwidth]{jardin01.png}
\label{Fig.sub.1.3.1}
}
\subfigure[Application sur le signal Jardin02]{
\includegraphics[width = 0.45\textwidth]{jardin02.png}
\label{Fig.sub.1.3.2}
}
\subfigure[Application sur le signal MarteauPiqueur]{
\includegraphics[width = 0.45\textwidth]{marteaaupiqueur01.png}
\label{Fig.sub.1.3.3}
}
\subfigure[Application sur le signal Ville01]{
\includegraphics[width = 0.45\textwidth]{ville01.png}
\label{Fig.sub.1.3.4}
}
\caption{Application sur les signaux donnés}
\label{Fig.main.2}
\end{figure}

Pour chacun des signaux, l'amplitude en fonction du temps est présentée en haut, tandis que la puissance en dBm en fonction du temps est affichée en bas. Les pointillés rouges désignent la puissance de référence à 8 dBm et nous observons bien que lorsque le signal dépasse 8 dBm et dure plus d'une seconde, nous avons une plage de son pénible (en rouge).
\\
\\
\begin{table}
  \centering
  \begin{tabular}{|c|c|c|c|c|}
    \hline
    & Ville01 & MarteauPiqueur01 & Jardin01 & Jardin02 \\
    \hline
    Puissance moyenne (dBm) &-27.363 & -22.7752 
  &-23.5662 & -21.5499\\
    \hline
    Puissance moyenne (mW) & 1.835 & 5.2781\times 10^-6 & 4.3992\times10^-6 & 6.9986\times10^-6\\
    \hline
    Durée & 43.5566 & 24.8176 & 21.058 & 29.0993\\
    \hline
    Tension RMS (V) & 0.0013547 & 0.0022974 & 0.0020974 & 0.0026455 \\
    \hline
  \end{tabular}
    \caption{Avant Amplification}
\end{table}
\\
\begin{table}
  \centering
  \begin{tabular}{|c|c|c|c|c|}
    \hline
    & Ville01 & MarteauPiqueur01 & Jardin01 & Jardin02 \\
    \hline
    Puissance moyenne (dBm) & 12.637 & 17.2248 & 16.4338 & 18.4501\\
    \hline
    Puissance moyenne (mW) & 0.018353 & 0.052781 & 0.043992& 0.069986\\
    \hline
    Durée (s)& 43.5566 & 24.8176 & 21.058 & 29.0993 \\
    \hline
    Tension RMS (V) & 0.13547 & 0.22974 & 0.20974 & 0.26455 \\
    \hline
  \end{tabular}
    \caption{Après Amplification}
\end{table}
\\
\newpage
% analyse des sons par rapport au ressenti à l'oreille sons pénible par rapport aux programme 
Dans ce tableau des caractéristiques des différents signaux, nous avons déterminé la puissance moyenne. Elle mesure la puissance d'un signal sur une période de temps donnée et elle est définie comme la quantité totale d'énergie transférée ou consommée divisée par la durée de cette période. Entre deux instants, elle est l'énergie dissipée par la durée d'observation et sa formule est : 
\begin{equation}
    P(x, t_{1}, t_{2}) = \frac{1}{t_{2} - t{1}} \int_{t_{1}}^{t_{2}}x_{c}(t)^2 dt
\end{equation}