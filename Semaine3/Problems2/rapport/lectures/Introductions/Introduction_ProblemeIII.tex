Nous envisageons la réalisation d'un enregistrement numérique en exploitant un microphone doté d'une sensibilité de S = -47 dBV et capable de capturer des niveaux sonores atteignant 130 dB SPL dans une plage de fréquence de 200 Hz à 20 kHz. En supposant l'absence totale de distorsion et de bruit introduit par le microphone, le signal analogique subit une amplification dont le gain, exprimé en décibels, doit être déterminé. La plage dynamique cible après amplification est fixée à [-1V,+1V]. Après cette étape, le signal est numérisé avec des paramètres spécifiques visant à garantir un rapport signal à bruit d'au moins 40 dB pour un signal acoustique de PdB SPL = 60 dB SPL. Nous entreprenons de déterminer le gain G nécessaire à l'amplification du signal analogique, de fournir un schéma fonctionnel détaillé pour la numérisation, de spécifier tous les paramètres requis, et de calculer le débit du signal numérique résultant. Enfin, nous examinerons la capacité de stockage nécessaire pour enregistrer une heure d'audio en stéréo, évaluerons la compatibilité des paramètres avec la qualité et le stockage sur CD, et analyserons leur aptitude à préserver le timbre des instruments d'un orchestre symphonique.
\newline
\\
\textbf{Problème III - Énoncé } \  \\
On souhaite effectuer un enregistrement numérique en utilisant un micro de sensibilité S = -47 dBV, de niveau sonore maximal de 130 dB SPL de bande passante 200 Hz - 20 kHz. On suppose que ce micro n'introduit aucune distorsion sur la bande considérée et aucun bruit. Le signal analogique est ensuite amplifié avec un gain G (en dB). La dynamique après amplification est [-1V,+1V]. Ce signal est ensuite numérisé. On choisit des paramètres de numérisation afin de garantir un rapport signal à bruit d'au moins 40 dB pour un signal acoustique de PdB SPL = 60 dB SPL.
\\
Déterminer le gain G à appliquer au signal analogique et donner le schéma fonctionnel de numérisation de ce signal, déterminer tous les paramètres et calculer le débit du signal numérique. 
\\
Quelle capacité de stockage faut-il pour enregistrer une heure d'audio en stéréo ? Les paramètres choisis sont-ils compatibles avec une qualité et un stockage sur CD ? Permettent-ils de respecter le timbre des instruments d'un orchestre symphonique ?
\\
Réfléchir sur l'utilité d'adopter un pas de quantification non pas uniforme, mais qui suit une loi logarithmique.