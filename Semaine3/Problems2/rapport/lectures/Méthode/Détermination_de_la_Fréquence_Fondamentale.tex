Le décodage des caractéristiques de chaque note s'effectue en calculant la puissance moyenne, la fréquence fondamentale ($f_0$), la fréquence de la plus haute harmonique ($f_h$), et le nombre d'harmoniques dans la bande de fréquences. Ces informations sont extraites à partir du signal audio, utilisant diverses fonctions telles que le calcul de la moyenne de puissance, la fréquence fondamentale par auto-corrélation, et le comptage des harmoniques. Les résultats sont affichés pour chaque note, fournissant une caractérisation détaillée des éléments musicaux contenus chaque fichier audio ainsi que la note et l’octave associé à la fréquence. 

L'ensemble de ces étapes contribue à élaborer un algorithme complet de détection et de caractérisation des notes de musique, avec une option facultative pour estimer la direction de la source sonore sur des signaux acquis par deux micros différents. 