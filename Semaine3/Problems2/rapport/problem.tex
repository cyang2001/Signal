
  \textbf{Problème II:} \  \\
  Proposer un algorithme qui permet de détecter, de caractériser une note de musique jouée par un instrument (violon, flûte, piano) et de déterminer sa hauteur. \\
  Cet algorithme devra produire les sorties suivantes: \\
  \begin{itemize}
    \item $t_d, t_f$: instants de début et de fin de note
    \item $P_dBm$: la puissance moyenne du signal en dBm
    \item $f_0$: la fréquence fondamentale et le nom et l'octave de la note jouée
    \item $f_h$: la fréquence haute telle que [0,$f_h$] contient 99.99\% de la puissance
    \item $n_h$: le nombre d'harmoniques dans cette bande de fréquences
    \item Toutes autres caractériques qui vous  paraît pertinente pour clssifier les notes par instrument de musique.
    On reprendra la méthode de détection d'un son utile du PBI et on justifiera les paramètres retenus.
  \end{itemize}

  
