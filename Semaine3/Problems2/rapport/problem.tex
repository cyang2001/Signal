
\textbf{Problème II :} \  \\
Proposer un algorithme qui permet de détecter, de caractériser une note de musique jouée par un instrument (violon, flûte, piano) et de déterminer sa hauteur. \\
Cet algorithme devra produire les sorties suivantes : \\
\begin{itemize}
    \item $t_d, t_f$: instants de début et de fin de note
    \item $P_dBm$: la puissance moyenne du signal en dBm
    \item $f_0$: la fréquence fondamentale et le nom et l'octave de la note jouée
    \item $f_h$: la fréquence haute telle que [0,$f_h$] contient 99.99\% de la puissance
    \item $n_h$: le nombre d'harmoniques dans cette bande de fréquences
    \item Toutes autres caractéristiques qui vous paraît pertinente pour classifier les notes par instrument de musique.
    On reprendra la méthode de détection d'un son utile du PBI et on justifiera les paramètres retenus.
\end{itemize}

\textbf{Problème IV :} \ \\
On souhaite détecter les signaux environnementaux qui comprennent des composantes spectrales très aiguës et pénibles pour l'oreille humaine. Ces signaux sont définis ainsi : 20\% de leur puissance est comprise dans les fréquences supérieures à 2 kHz, avec une puissance sonore totale supérieure à 110 dB SLP. \\
Concevoir un système de détection de ces signaux pénibles fondé sur filtrage
numérique. On prendra un micro de sensibilité égale à -67 dBV et de gain égal à 16 dB. Discuter les critères énoncés.

  